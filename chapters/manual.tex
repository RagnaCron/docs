%%
%% Copyright (C) 2020 by Manuel Werder
%%
%% This work may be distributed and/or modified under the
%% conditions of the LaTeX Project Public License, either version 1.3
%% of this license or (at your option) any later version.
%% The latest version of this license is in
%%
%%     http://www.latex-project.org/lppl.txt
%%
%! Date = 13.05.20
\mode*

\section{Einleitung}
\label{sec:einleitung}
\mode<article>{\setcounter{page}{1}}

\begin{frame}[fragile]
    \frametitle<presentation>{Einleitung}
    some text\ldots
    \mode<article>{\vSpaceStyle{}}
    some more text
\end{frame}


\subsection{Vorgaben}
\label{subsec:vorgaben}
Diese werden wie folgt aufgelistet:
\begin{frame}[fragile]
    \frametitle<presentation>{Voragaben}

    \begin{itemize}
        \item one
        \item two three
    \end{itemize}
\end{frame}


\subsection{Modulidentifikation}
\label{subsec:modulid}
Die Handlungsziele gemäss Modulidentifikation:
\begin{frame}[fragile]
    \frametitle<presentation>{Modulidentifikation}
    \begin{enumerate}
        \item one
        \item two
    \end{enumerate}
\end{frame}


\subsection{Projektübersicht}\label{subsec:projekt}
\begin{frame}[fragile]
    \frametitle<presentation>{Projektübersicht}
    Es werden hier die wichtigsten Projekttools wie die Beteiligten aufgezeigt.
\end{frame}

\begin{frame}[fragile]
    \frametitle<presentation>{Beteiligte}
    \begin{table}[h!]
        \centering
        \begin{tabularx}{0.8\textwidth} {
        | >{\raggedright\arraybackslash}X
        | >{\raggedright\arraybackslash}X | }
            \hline
            Student und Projektleiter & Manuel Werder \\
            \hline
%        Dozent & \sffamily   \\
%        \hline
        \end{tabularx}
        \caption{Projektbeteiligte}
        \label{tab:1}
    \end{table}
\end{frame}

\begin{frame}[fragile]
    \frametitle<presentation>{Software, Tools und Services}
    \begin{table}[h!]
        \centering
        \begin{tabularx}{0.8\textwidth} {
        | >{\raggedright\arraybackslash}X
        | >{\raggedright\arraybackslash}X | }
            \hline
            \LaTeX & \href{https://www.overleaf.com}{Overleaf} \\
            \hline
        \end{tabularx}
        \caption{Software, Tools und Services}
        \label{tab:2}
    \end{table}
\end{frame}



\section{Konzepte}\label{sec:konzepte}
\begin{frame}[fragile]
    \frametitle<presentation>{Konzepte}
    Deine Konzepte\ldots
\end{frame}
%\figureStyle{width=0.25\textwidth}{GameScreen}{Game Screen}{fig:GameScreen}



\section{Selbstständigkeitserklärung}\label{sec:ssk}
\begin{frame}[fragile]
    \frametitle<presentation>{Selbstständigkeitserklärung}
    Hiermit erkläre ich, dass dieses Dokument selbstständig verfasst wurde, dass alle Angaben
    überprüft und entsprechend korrekt referenziert sind.
    \mode<article>{\vSpaceStyle{}}
    \gitAuthorDate~\gitAuthorName
\end{frame}


