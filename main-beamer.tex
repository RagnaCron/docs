%! Author = thor
%! Date = 15.06.20

% Preamble
\documentclass[12pt,aspectratio=169]{beamer}

\RequirePackage{setmanu-styles}
\RequirePackage{hyperref}
\hypersetup{
colorlinks  = true,
linkcolor   = blue,
filecolor   = magenta,
urlcolor    = blue,
}
\RequirePackage{tabularx}

\usetheme[progressbar=frametitle]{metropolis}

% ----------------------------------

\title{Title}
\subtitle{Subtitle}

\date{\gitAuthorDate}
\author{\gitAuthorName, \gitAuthorEmail}
\institute{CSBE\\
\vfill\color{gray}\tiny\sffamily$[$git$]$\gitStyler}

% ----------------------------------

% Document
\begin{document}

    \maketitle

    % Theme "metropolis" supports 4 different titleformats
    %   frame=regular
    %   frame=smallcaps
    %   frame=allsmallcaps
    %   frame=allcaps
    \metroset{titleformat frame=smallcaps}

    % In Abhängigkeit vom Umfang der einzelnen Kapitel werden diese
    % so (pro Seite) gruppiert, dass jede Seite jeweils einen kompletten
    % Schulungstag abdeckt.
    %
    % Dies kann mit folgender Option erreicht werden.
    %
    % \tableofcontents[sections={1-5}]

    \begin{frame}{Inhaltsverzeichnis}
        \centering
        \setbeamertemplate{section in toc}[sections numbered]
        \begin{minipage}[t]{0.5\textwidth}
            \tableofcontents[hideallsubsections,sections={1-6}]
        \end{minipage}
%        \begin{minipage}[t]{0.4\textwidth}
%            \tableofcontents[hideallsubsections,sections={4-6}]
%        \end{minipage}
    \end{frame}

    %%
%% Copyright (C) 2020 by Manuel Werder
%%
%% This work may be distributed and/or modified under the
%% conditions of the LaTeX Project Public License, either version 1.3
%% of this license or (at your option) any later version.
%% The latest version of this license is in
%%
%%     http://www.latex-project.org/lppl.txt
%%
%! Date = 13.05.20
\mode*

\secStyle{Einleitung}\label{sec:einleitung}
\mode<article>{\setcounter{page}{1}}
some text...
\vSpaceStyle{}
some text

\subSecStyle{Vorgaben}\label{subsec:vorgaben}
Diese werden wie folgt aufgelistet:

\begin{itemize}
    \item one
\end{itemize}


\subSecStyle{Modulidentifikation}\label{subsec:modulid}
Die Handlungsziele gemäss Modulidentifikation:

\begin{enumerate}
    \item one
\end{enumerate}


\subSecStyle{Projektübersicht}\label{subsec:projekt}
Es werden hier die wichtigsten Projekttools die wie Beteiligten aufgezeigt.

\begin{table}[h!]
    \centering
    \begin{tabularx}{0.8\textwidth} {
    | >{\raggedright\arraybackslash}X
    | >{\raggedright\arraybackslash}X | }
        \hline
        Student und Projektleiter & Manuel Werder \\
        \hline
%        Dozent & \sffamily   \\
%        \hline
    \end{tabularx}
    \caption{Projektbeteiligte}
    \label{tab:1}
\end{table}

\begin{table}[h!]
    \centering
    \begin{tabularx}{0.8\textwidth} {
    | >{\raggedright\arraybackslash}X
    | >{\raggedright\arraybackslash}X | }
        \hline
        \LaTeX & \href{https://www.overleaf.com}{Overleaf} \\
        \hline
    \end{tabularx}
    \caption{Tools, Software und Services}
    \label{tab:2}
\end{table}



\secStyle{Konzepte}\label{sec:konzepte}
%\figureStyle{width=0.25\textwidth}{GameScreen}{Game Screen}{fig:GameScreen}



\secStyle{Selbstständigkeitserklärung}\label{sec:ssk}
Hiermit erkläre ich, dass dieses Dokument selbstständig verfasst wurde, dass alle Angaben korrekt
und überprüft sind und entsprechend korrekt referenziert wird.
\vSpaceStyle{}
\gitAuthorDate~\gitAuthorName


\end{document}